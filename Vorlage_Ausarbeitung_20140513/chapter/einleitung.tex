%% Einleitung.tex
%% $Id: einleitung.tex 28 2007-01-18 16:31:32Z bless $
%%

\chapter{Einleitung}
\label{ch:Einleitung}
%% ==============================
Hinweis: In die Einleitung gehört die Motivation 
und Einleitung in die Problemstellung. Die Problemstellung
kann in der Analyse noch detaillierter beschrieben werden.

Bla fasel\ldots

Heutzutage gibt es viele Geräte die einen Vibrationsmotor besitzen, sei es das Handy, Smartwatches, Fitnessarmbänder (o.ä.). Bei meiner Recherche, gab es bis bisher keine wirkliche Umsetzung um Vibrationen für eine Person personalisieren zu lassen, mit ein paar kleinen ausnahmen, auf die ich noch zu sprechen komme. Im Folgenden habe ich ein paar Geräte angesprochen, die einen Ansatz von Personalisierung von Vibrationen haben.

Bei der Apple Watch beispielsweise gibt es eine von Apple selbst entworfene Taptic Engine. Der Name bildet sich aus dem Wort "Taktil" und "Haptisch" zusammen. Diese Taptic Engine erzeugt eine mechanische Rückmeldung, die bei den meisten Apple Produkten bereits verbaut ist. Somit wird kein Vibrationsmotor mehr verwendet sondern die beschriebene Taptic Engine. Trotz neu erfinden einer mechanischen Rückmeldung kann man die Apple Watch nicht personalisieren wie lang die Rückmeldungen jetzt sein sollen. Das einzige was man Einstellen kann, ist die Stärke der Uhr. Diese ist in 3 Stärkestufen unterteilt. Dabei kann man aber nicht von Personalisierung sprechen. 

Aktuell ist es so, dass man bei dem Hersteller Samsung keinerlei Möglichkeit hat sich dort die Vibrationen zu Personalisieren. Man kann hier lediglich unter einer Handvoll vordefinierten Vibrationsmuster entscheiden. 

Beim iPhone vom Apple, kann man unter Einschränkungen sich wirklich mal ein eigenes Muster erstellen. Dabei geht man in die jeweilige Einstellungen hinein und es erscheint ein graues Bild. Beim drücken auf dem Display wird an der Stelle eine Vibration erzeugt. Dabei hat man inetwa 10 Sekunden Zeit um sich ein eigenes Muster zu definieren, indem man auf den Bildschirm drückt. Diese Vibrationsmuster sind jedoch nur Eingeschränkt nutzbar. Zum Beispiel lassen sich diese Muster nur unter Klingeltönen, dem Nachrichtenton, Errinerungshinweisen, (o.ä.) hinzufügen. Das was jedoch fehlt ist, dass man diese persönlichen Muster auch zu einer bestimmten Benachrichtigung einer Application abspielen lassen könnte. Das heißt, ich kann nicht unterscheiden, welche Benachrichtigung ich erhalte, wenn ich mein Handy in der Hosentasche habe wenn es auf Stumm gestellt ist. 

Das letzte Gerät was ich ansprechen möchte, ist von keiner großen Firma wie Apple oder Samsung sondern ein StartUp namens Martian. Diese Firma hat eine Uhr hergestellt, die man mittels einer App auf dem Handy anpassen kann. Mittels der Application kann man sich für mehrere Hunderte Apps die Benachrichtigungen senden, ein Muster mit bis zu 4 Signalen auf der Uhr als Vibration darstellen lassen. Die Signale sind zwischen Lang, Kurz und Pause (also kein Signal) Darstellbar. 
Die Uhr hat mich als einziges Produkt überzeugt, dass man sich persönliche Vibrationsmuster erstellen und abspielen lassen kann. Natürlich ist die Länge und Stärke der Vibration damit vorgegeben.

Somit wollte ich herausfinden, ob es möglich ist für jedes Individuum eine eigene passende Länge und Stärke von Vibrationssignalen zu erstellen, so dass die Kombination der Signale noch erkannt werden.
 

%% ==============================
\section{Zielsetzung der Arbeit}
%% ==============================
\label{ch:Einleitung:sec:Zielsetzung}

Was ist die Aufgabe der Arbeit? 

Bla \ldots
Die Aufgabe besteht darin, für ein Individuum mit einem Programm drei verschiedene Vibrationssignale für Ihn so anzupassen, dass die Werte von Ihm besser erkannt werden als vorgegebene Werte. Dabei wird zur Bestimmung eines Vibrationssignals ein Evolutionärer Algorithmus verwendet. Nachdem dieser passende Wert gefunden worden ist, wird überprüft, wie gut im Vergleich zu Vorgegebenen Werten die Daten erkannt werden.
Grenzen bestimmen,
Algorithmus,
Erkennen,
Studie durchführen


%% ==============================
\section{Gliederung der Arbeit}
%% ==============================
\label{ch:Einleitung:sec:Gliederung}

Was enthalten die weiteren Kapitel?

Bla fasel\ldots

%%% Local Variables: 
%%% mode: latex
%%% TeX-master: "diplarb"
%%% End: 
