%% grundlagen.tex
%% $Id: grundlagen.tex 28 2007-01-18 16:31:32Z bless $
%%

\chapter{Grundlagen}
\label{ch:Grundlagen}
%% ==============================
Die Grundlagen müssen soweit beschrieben
werden, dass ein Leser das Problem und
die Probleml{\"o}sung  versteht.Um nicht zuviel 
zu beschreiben, kann man das auch erst gegen 
Ende der Arbeit schreiben.

Bla fasel\ldots

%% ==============================
\section{Tactile Ger{\"a}te}
%% ==============================
\label{ch:Grundlagen:sec:Taktile Ger{\"a}te}

Bla fasel\ldots

%% taktile geräate.tex
%% $Id: taktilegeraete.tex 28 2007-01-18 16:31:32Z bless $
%%

Ein Taktiles Gerät ist ein Gerät, dass Informationen an einen Menschen mitteilen möchte, dies geschieht durch die Wahrnehmung der Haut. 
Ein Taktiles Gerät ist ein Gerät, dass Informationen durch die Wahrnehmung des Menschen mittels Körperkontakt darstellet.  \cite{gemperle2001design}

Tatile Geräte werden heutzutage öfter verwendet, als man es eigentlich wahrnimmt. 
Ein einfaches Beispiel ist das Handy. 
Eine Person hat sein eigenes Handy in der Hosentasche. Bei einer eingehenden Nachricht, muss der Benutzer mitgeteilt werden, dass eine Nachricht empfangen wurde.
Dies geschieht normalerweise mit dem Klingelton. Wenn man jetzt beschäftigt ist und nicht durch ein lautes klingeln gestört werden möchte, dann stellt man den Ton ab. 
Um dennoch darauf Aufmerksam zu machen, dass eine Nachricht eingetroffen ist, nutzt man die Vibration des Handys.

Das Handy ist nur eins von vielen Beispielen, was man über den Alltag noch für Taktile Geräte verwendet.

//Man nutzt Vibrationen um darauf aufmerksam zu machen, dass tzotz abgeschaltetem Klingelton, eine Nachricht empfangen wurde. 

//Falls eine Person das Handy in der Hosentasche haben sollte und eine eingehende Nachricht empfangen wurde, will das Handy dem Benutzer das mitteilen.
Dies passiert im Normalfall durch einen Klingelton. Um als Taktiles Gerät definiert zu werden, muss es dem Benutzer per Körperkontakt mitteilen, dass eine Nachricht eingegangen ist. 
Das geschieht nicht durch den Klingelton sondern durch Vibration. 


%% ==============================
\section{Genetischer Algorithmus}
%% ==============================
\label{ch:Grundlagen:sec:Genetischer Algorithmus}

Bla fasel\ldots
%% Genetischer Algorithmus.tex
%% $Id: Genetischer Algorithmus.tex 28 2007-01-18 16:31:32Z bless $
%%

Die Evalutionären Algorithmen sind stochastische Optimierungsverfahren. Man findet damit nicht die beste Lösung für ein Problem, jedoch findet man eine gute Annäherung.
Dabei hat man sich bei den Evolutionären Algorithmen an der Biologischen Evolution von Darwin inspirieren lassen. \cite{selzam2003genetische}

In der Biologie ist jeder lebende Organismus ein Induvidiuum.
Jedes Induvidiuum besitzt Erbinformationen in der Form von Chromosomen. Die Erbinformationen werden auch \textbf{Gene} oder \textbf{DNA} genannt.
Eine Gruppe von Induviduen wird als \textbf{Population} bezeichnet.

Der Algorithmus wurde anhand der folgenden Merkmale erstellt.

%% ==============================
\subsection{Selektion}
%% ==============================
\label{ch:Grundlagen:sec:Taktile Geräte:subsec:Selektion}

Eine \textbf{Selektion} ist ein Mechanismus, bei dem man sich zwei Individuen auswählt, um anschließend die Gene der beiden Individuen, mittels der sogenannten \textbf{Rekombination}, zu kombinieren. Dabei gibt es verschiedene Selektionsstrategien. Man versucht die Individuen zu finden, um eine bestmöglichste Lösung für ein Problem zu liefern.

%% ==============================
\subsection{Variation}
%% ==============================
\label{ch:Grundlagen:sec:Taktile Geräte:subsec:Variation}

Man sollte eine zahlreiche Variation von Genen der einzelnen Individuen besitzen. 
Denn nehme anhand einem Beispiel von Süßigkeiten einmal kurz an, dass alle Süßigkeiten die gleiche Farbe und die gleiche Form haben. Wenn man sich jetzt zwei zufällige Süßigkeiten auswählen würde, hätten Sie keinerlei unterschiede und so würde das Nachkommen die gleichen Gene besitzen. Damit genau das nicht auftaucht, versucht man zu Beginn eine große Variation an Individuen zu erzeugen und diese als Anfangs Population für einen Evolutionären Algorithmus zu nutzen. Mittels Rekombination und Mutation wird dabei ein neues Induvidiuum für die nächste Generation erzeugt. 


//Durch Rekombination und Mutation der DNA der Induviduuen erhält man eine Variation. 
//Die Rekombination und Mutation erzeugt lediglich ein neues Induviduum aus der DNA der zuvor selektierten Induviduen.

Um die nächsten Generationen zu bilden wird eine Rekombination und Mutation auf die DNA der Selektierten Induviduen ausgeführt. 
Die Variation der DNA  spielt eine wichtige Rolle, denn die ist für die nachfolgende Rekombination und Mutation entscheident für die nächsten Generationen. 
Um bei der Selektion unterschiedliche Induviduuen ausgewählt werden können, benötigt man zunächst eine Variation 

%% ==============================
\subsection{Vererbung / Gendrift}
%% ==============================
\label{ch:Grundlagen:sec:Taktile Geräte:subsec:Variation}
Bla fasel \dots

%% ==============================
\subsection{Allgeimeiner Vorgang eines Evolutionären Algorithmus}
%% ==============================
\label{ch:Grundlagen:sec:Taktile Geräte:sec:Allgeimeiner Vorgang eines Evolutionären Algorithmus}
Ein Evolutionärer Algorithmus besitzt grundsätzlich immer die gleichen Komponenten die miteinander.

%% ==============================
\subsection{Initialisierung}
%% ==============================
\label{ch:Grundlagen:sec:Initialisierung}
Man erzeugt sich zur Initialisierung eine Population von Individuen, die eine Variation von Genen besitzt. 

%% ==============================
\subsection{Bewertung der Individuen}
%% ==============================
\label{ch:Grundlagen:sec:Bewertung der Individuen}
Bevor man eine neue Population für die nächste Generation berechnen kann, muss man zuerst mithilfe einer Fitnessfunktion jedes Induvidiuum einen Fitnesswert bestimmen. 

%% ==============================
\subsection{Selektion}
%% ==============================
\label{ch:Grundlagen:sec:Selektion}
Anhand dem Fitnesswert der Induviduen werden zwei zufällige Induviduen bestimmt. Diese selektierten Induviduen werden als Eltern für die Nächste Generation benutzt.
Bei der Selektion wird ein hoher Fitnesswert bevorzugt.

%% ==============================
\subsection{Rekombination}
%% ==============================
\label{ch:Grundlagen:sec:Rekombination}
Die Gene der ausgewählten Eltern werden miteinander kombiniert und bilden die Gene des neuen Induvidiuum für die Population der nächsten Generation. Hier gibt es verschiedene Kombinationsmöglichkeiten, die angewendet werden können.

%% ==============================
\subsection{Mutation}
%% ==============================
\label{ch:Grundlagen:sec:Mutation}
Bei dem erzeugen Induvidiuum besteht eine Chance, dass die kombinierten Gene durch eine Mutation verändert werden. 

%% ==============================
\subsection{Wiederholung durch neuer Generation}
%% ==============================
\label{ch:Grundlagen:sec:Wiederholung durch neuer Generation}
Der Vorgang der Selektion, Rekombination und Mutation wird so oft ausgeführt, bis man eine neue Population hat, die genauso groß ist, wie die Anfangapopulation.
Nachdem die neue Population erzeugt wurde, wird diese durch die Anfangspopulation ersetzt und man führt den Algorithmus erneut aus. 
Dies geschieht so lange, bis man eine hinreichende Lösung für das Problem hat. 




Ich habe meinen Evolutionären Algorithmus so angepasst, dass bei mir ein Induviduum ein Signal ist. 
Ich habe dem Benutzer das Signal mit dem Wearable abspielen lassen und im Anschluss Fragen beantworten lassen. 
Er sollte bewerten wie gut er das Signal erkannt hat. Die Bewertung vom Benutzer war entscheidend um nach der kompletten Bewertung der Population 

%% ==============================
\section{Verwandte Arbeiten}
%% ==============================
\label{ch:Grundlagen:sec:RelatedWork}
Hier kommt "`Related Work"' rein.
Eine Literaturrecherche sollte so vollst{\"a}ndig wie m{\"o}glich sein,
relevante Ans{\"a}tze m{\"u}ssen beschrieben werden und es sollte deutlich 
gemacht werden, wo diese Ans{\"a}tze Defizite aufweisen oder nicht
anwendbar sind, z.\,B. weil sie von anderen Umgebungen oder 
Voraussetzungen ausgehen.

Bla fasel\ldots

%% custom vibs.tex
%% $Id: Genetischer Algorithmus.tex 28 2007-01-18 16:31:32Z bless $
%%

Da heutzutage beinahe jedes Ger�t ein Vibtarionsmotor verbaut hat, sei es das Handy, die Smartwatches oder Fitnessarmb�nder (uvm.), werde ich im folgenden noch auf einige aktuelle Technologien und deren Umsetzung der personalisierten Vibtationsmuster zu sprechen kommen. 

%% ==============================
\subsection{Taptic Engine}
%% ==============================
\label{ch:Grundlagen:sec:RelatedWork:subsec:TapticEngine}

Die Taptic Engine ist eine von der Firma Apple eine eigen entwickeltes Vibrationsmotor, dass heutzutage in nahezu allen Apple Produkten vertreten ist. Das erste Ger�t was die Taptic Engine bekommen hatte, war die Apple Watch. Der Name bildet sich aus dem Wort "Taktil" und "Haptisch". 
Trotz des neu erfinden einer mechanischen R�ckmeldung bietet Apple keine Personalisierung, wie lange eine R�ckmeldung erfolgen soll, f�r die Apple Watch an. Die Einstellungsm�glichkeiten an der Apple Watch ist lediglich die St�rke der Vibration. Diese ist in 3 Stärkestufen unterteilt. Dabei kann man aber nicht wirklich von einer Personalisierung sprechen. 

(BILD) 

Aktuell ist es so, dass man bei dem Hersteller Samsung keinerlei Möglichkeit hat sich dort die Vibrationen zu Personalisieren. Man kann hier lediglich unter einer Handvoll vordefinierten Vibrationsmuster entscheiden. 
 
%% ==============================
\subsection{Personalisierte Vibration}
%% ==============================
\label{ch:Grundlagen:sec:RelatedWork:subsec:PersonalisierteVibration}

(BILD)

Der Hersteller Apple hat auch bei dem iPhone eine M�glichkeit geboten eigene Vibrationsmuster zu erstellen, jedoch mit Einschr�nkungen.
Wenn man in die jeweilige Einstellung der iPhones gelangt, erscheint das folgende Bild. Beim dr�cken auf das Display wird an der Stelle eine Vibration erzeugt. Man hat 10 Sekunden um ein eigenes Muster zu erzeugen, indem man wiederholt auf den Bildschirm dr�ckt. An der Stelle, an der man den Bildschirm ber�hrt hat, erscheint visuell um der Position ein Kreis. Die erzeugten Vibrationen werden in einer Leiste visuell angezeigt. 
Man kann sich beliebig viele Vibrationsmuster speichern, die bis zu 10 Sekunden lang sind. \cite{fleizach2016custom}

Die Einschr�nkung die man hier nennen muss ist, dass man die Vibrationsmuster nur f�r Systeminterne Funktionen benutzen kann. Dies bedeutet, dass man die Funktionen f�r Klingelt�ne, Nachrichtent�ne, Erinnerungshinweise, Kalenderhinweise (o.�.) hinzuf�gen kann. F�r eine andere Anwendung, die nicht im Betriebsystem integriert ist, ist das nicht m�glich. Somit k�nnen Benachrichtigungen von anderen Entwicklern keine eigene Vibrationsmuster zuweisen. Somit kann man bei Benachrichtigungen von anderen Entwicklern nicht anhand der Vibrationen des iPhones  unterscheiden.

%% ==============================
\subsection{Personalisierte Smartwatch}
%% ==============================
\label{ch:Grundlagen:sec:RelatedWork:subsec:PersonalisierteSmartwatch}

Das letzte Ger�t ist eine Smartwatch von einem kleinen StartUp namens Martian. 

Das letzte Gerät was ich ansprechen möchte, ist von keiner großen Firma wie Apple oder Samsung sondern ein StartUp namens Martian. Diese Firma hat eine Uhr hergestellt, die man mittels einer App auf dem Handy anpassen kann. Mittels der Application kann man sich für mehrere Hunderte Apps die Benachrichtigungen senden, ein Muster mit bis zu 4 Signalen auf der Uhr als Vibration darstellen lassen. Die Signale sind zwischen Lang, Kurz und Pause (also kein Signal) Darstellbar. 
Die Uhr hat mich als einziges Produkt überzeugt, dass man sich persönliche Vibrationsmuster erstellen und abspielen lassen kann. Natürlich ist die Länge und Stärke der Vibration damit vorgegeben.

Somit wollte ich herausfinden, ob es möglich ist für jedes Individuum eine eigene passende Länge und Stärke von Vibrationssignalen zu erstellen, so dass die Kombination der Signale noch erkannt werden.

Die Hypothese die ich mit dieser Bachelorarbeit beantworten möchte ist, ob man mittels personalisierten Vibrationen eine Folge von Vibrationen besser unterscheiden als vorgegebene Vibrationen.
Um dies beantworten zu können verwende ich Wearable zur Darstellung der Vibrationssignale und einen Genetischen Algorithmus um die personalisierten Vibrationen zu ermitteln.




%%% Local Variables: 
%%% mode: latex
%%% TeX-master: "diplarb"
%%% End: 
