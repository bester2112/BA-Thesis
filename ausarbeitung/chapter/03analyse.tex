%% analyse.tex
%% $Id: analyse.tex 28 2007-01-18 16:31:32Z bless $

\chapter{Analyse}
\label{ch:Analyse}
%% ==============================
In diesem Kapitel sollten zun{\"a}chst das zu l{\"o}sende Problem
sowie die Anforderungen und die Randbedingungen 
einer L{\"o}sung\index{L{\"o}sung} beschrieben werden (also nochmal
eine pr{\"a}zisierte Aufgabenstellung\index{Aufgabenstellung}).

Dann folgt {\"U}blicherweise ein {\"U}berblick {\"u}ber bereits existierende
L{\"o}sungen bzw. Ans{\"a}tze, die meistens andere Voraussetzungen bzw.
Randbedingungen annehmen.


%% ==============================
\section{Anforderungen}
%% ==============================
\label{ch:Analyse:sec:Anforderungen}
Anforderungen und Randbedingungen\index{Randbedingungen} \ldots

%% Anfroderungen.tex
%% $Id: Anfroderungen.tex 28 2007-01-18 16:31:32Z bless $

Die Aufgabenstellung bestand darin, herauszufinden, ob ein personalisiertes Vibrationsmuster besser als ein vordefiniertes Vibrationsmuster erkannt wird.
Um dies zu lösen musste man sich im Vorfeld ein paar Gedanken über die Repräsenatation eines Signals, die Dekodierung und über die Übertragung machen.
Im Folgenden werden diese Entscheidungsfindungen beschrieben.

Die Übertragung

Zu aller erst hat man sich das vorgegebene Armband genauer inspiziert. Dabei konnte man das Armband über eine Serielle Schnittstelle sowie über Bluetooth Low Energie (LE) nutzen. 
Bei der Nutzung der Seriellen Schnittstelle hatte man zwar den Vorteil, dass man sich nicht mittels Bluetooth auseinander setzen müsse, jedoch ist man an einem Kabel über den PC verbunden gewesen, was zur Einschränkung der Bewegungsfreiheit \dots hat. Um genau diesen Nachteil nicht zu haben hat man sich für Bluetooth LE entschieden. Dabei gab es auch einige Nachteile, die man beheben musste. Die Kommunikation mittels Bluetooth LE hatte eine maximale Datenübertragung von 20 Bytes. Diese wurden jeweils in zwei Bytes Blöcke unterteilt.

Anhand der Begrenzung der Datenübertragung hat man sich eine geeignete Dekodierung des Signals überlegen müssen. 











\dots
Dabei musste man sich definieren wie ein Signal aufgebaut gewesen ist. Dabei musste für ein Signal eine Datenstruktur erstellt werden. 



Dekodierung eines Signals.

Evolutionärer Algorithmus musste erstellt werden mit allen komponenten, Population Fitness funktion, usw

Kommunikation mit dem Armband musste aufgebaut werden. 

Es sollte eine Studie ausgeführt werden um das System zu testen 

Dabei sollte eine Grafische Oberfläche erstellt werden, damit der Benutzer selbst eine Eingabe in den PC machen konnte um nächste Signale abspielen zu können.

Es sollten die Daten ausgewertet werden.



%% ==============================
\section{Existierende L{\"o}sungsans{\"a}tze}
%% ==============================
\label{ch:Analyse:sec:RelatedWork}

Hier kommt eine ausf{\"u}hrliche Diskussion
von "`Related Work"'.


%% Existierende L{\"o}sungsans{\"a}tze.tex
%% $Id: Existierende L{\"o}sungsans{\"a}tze.tex 28 2007-01-18 16:31:32Z bless $

%existierende Lösungsansätze müssten im verglichen werden und überprüft werden 
%ob mein ansatz für die eine Lösung bringt oder anders herum


%% ==============================
\section{Weiterer Abschnitt}
%% ==============================
\label{ch:Analyse:sec:Abschnitt}

Bla fasel\ldots hat auch schon \cite{TB2000} gesagt und
\cite{TB98,JSAC96,qosr} sollte man mal gelesen haben.
Abbildung~\ref{fig:test} auf S.~\pageref{fig:test} sollte man
sich mal anschauen.

%% Weiterer Abschnitt.tex
%% $Id: WeitererAbschnitt.tex 28 2007-01-18 16:31:32Z bless $

% System aufbau / ausf{\"u}hrung der Studie / analyse aufbau
% beschreibt, was ich gemacht habe.

\paragraph{R{\"u}ttelflug}

%Das vorgegebene Wearable von TECO ist ein {\"a}hnliches  
Aus dem Paper von \cite{pescara2016ruttelflug} wurde ein Wearable f{\"u}r das Paragliding entworfen.
Beim Entwurf des Wearables hat man sich an den vorhin genannten Anforderungen von \cite{gemperle2001design} gehalten.
Das Wearable besitzt einen Traktor oberhalb und einen unterhalb Handgelenks und wurde sowohl in einer Laborstudie als auch in einer Feldstudie getestet.
%Die daraus ergebenen Ergebnissen  
F{\"u}r die Laborstudie wurden Vibrationssignale in der L{\"a}nge von 100, 200, 400, und 800ms abgespielt. 
Jeder Proband hat dieselbe Reihenfolge von Aufgaben ausgef{\"u}hrt bekommen. 
Eine der Aufgaben der Probanden war den Namen und die L{\"a}nge des Signals zuzuordnen. 
Aus der Laborstudie hat sich ergeben, dass die Probanden die Signall{\"a}nge 800ms mit einer Genauigkeit von 97,6\% sowie die beiden Singnalsl{\"a}nge 200 und 400ms zu 100\% richtig erkannt haben. 

Die generischen Werte f{\"u}r die Erkennung der Muster hat man aus dem Paper {\"u}bernommen.

Blindtext Blindtext Blindtext Blindtext Blindtext Blindtext Blindtext
Blindtext Blindtext Blindtext\index{Blindtext} Blindtext Blindtext Blindtext Blindtext

\begin{figure}[!htbp]
  \centering
  \fbox{\parbox{0.8\textwidth}{
  Abbildungen sollten m{\"o}glichst als EPS (Encapsulated Postscript) 
  bzw. PDF eingebunden werden.
  Zur Erzeugung sauberer EPS-Dateien empfiehlt sich das Tool \texttt{ps2eps}
  zur Nachbearbeitung von Postscript-Dateien. Mit \texttt{epstopdf} kann
  dann eine PDF-Datei zum Einbinden erzeugt werden.}}
  \caption{Testabbildung}
  \label{fig:test}
\end{figure}

Blindtext Blindtext Blindtext Blindtext Blindtext Blindtext Blindtext
%% ==============================
\section{Zusammenfassung}
%% ==============================
\label{ch:Analyse:sec:zusammenfassung}

Am Ende sollten ggf. die wichtigsten Ergebnisse nochmal in \emph{einem}
kurzen Absatz zusammengefasst werden.

%% Zusammenfassung.tex
%% $Id: Zusammenfassung.tex 28 2007-01-18 16:31:32Z bless $




%%% Local Variables: 
%%% mode: latex
%%% TeX-master: "diplarb"
%%% End: 
