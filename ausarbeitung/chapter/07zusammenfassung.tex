%% zusammenf.tex
%% $Id: zusammenf.tex 4 2005-10-10 20:51:21Z bless $
%%

\chapter{Zusammenfassung und Ausblick}
\label{ch:Zusammenfassung}
%% ==============================

%Am Ende sollten ggf. die wichtigsten Ergebnisse nochmal in \emph{einem}
%kurzen Absatz zusammengefasst werden.

Die Hypothese "`personalisierte Vibrationen weden besser als vorgegebene Vibrationen erkannt"' wird im Folgenden beantwortet.
%Die Hypothese, die ich im Folgenden beantworten werde, lautet, dass die personalisierten Vibrationen besser als die vorgegebenen Vibrationen erkannt werden.
Im Rahmen dieser Bachelorarbeit wurde ein System entwickelt zur Bestimmung von personalisierten Vibrationsmustern. 
Für das System habe man die Komponenten für einen Evolutionären Algorithmus entworfen und implementiert.
Innerhalb von vier Iterationen habe man 32 Probanden das System evaluieren lassen. 
Die Ergebnisse zeigen, dass die personalisierten Vibrationssignale besser erkannt worden sind als die generuschen Vibrationssignale.

Man habe für diese Bachelorarbeit einen genetischen Algorithmus entickelt und implentiert, 



Man darf auch nicht unerwähnt lasse, dass die Probanden niemals eine Rückmeldung erhalten haben, dass Sie das Signal richtig oder Falsch bewertet hatten.
%Die Probanden haben niemals eine R{\"u}ckmeldung dar{\"u}ber erhalten, ob Sie richtig oder falsch liefen.


Man kann feststellen, das je mehr Signale man in ein Muster packt, desto höher wird die Wahrscheinlichkeit, dass die Muster nicht korrekt erkannt werden. 


genetische Signale wurden besser erkannt
die 4 Iterationen hatten keine Alzugroßen auswirkungen auf die Stimmung
zwischen Kurz und Mittel / Mittel und Kurz schwer erkennbar.
mittelwerte sind bei manchen probanden oft die personalisierten were gewesen

je höher die komlexität desto schlechter die genauigkeit



Da die Studie nur mit einer Anzahl von 32 Probanden durchgef{\"u}hrt wurde, kann man diese Ergebnisse jedoch nur mit Vorbehalt betrachten, da es diese Ergebnisse nicht einfach auf eine gr{\"o}{\ss}ere Anzahl an Personen schlie{\ss}en kann. 
Somit hei{\ss}t es nicht, dass Personen die weiblich sind schlechter Signale erkennen als m{\"a}nnliche Personen. Wenn man dies {\"u}berpr{\"u}fen wollen w{\"u}rde, m{\"u}sste man die Studie auf eine gr{\"o}{\ss}ere Anzahl an Probanden erweitern. Diese Werte sind nur die Ergebnisse die aus meiner Studie dabei herausgekommen sind.







GLOSSAR

Force Feedback:

%%% Local Variables: 
%%% mode: latex
%%% TeX-master: "diplarb"
%%% End: 
