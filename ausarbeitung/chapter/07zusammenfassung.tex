%% zusammenf.tex
%% $Id: zusammenf.tex 4 2005-10-10 20:51:21Z bless $
%%

\chapter{Zusammenfassung und Ausblick}
\label{ch:Zusammenfassung}
%% ==============================

%Am Ende sollten ggf. die wichtigsten Ergebnisse nochmal in \emph{einem}
%kurzen Absatz zusammengefasst werden.

Die Hypothese "`personalisierte Vibrationen werden besser als vorgegebene Vibrationen erkannt"' wird im Folgenden beantwortet.
%Die Hypothese, die ich im Folgenden beantworten werde, lautet, dass die personalisierten Vibrationen besser als die vorgegebenen Vibrationen erkannt werden.
Im Rahmen dieser Bachelorarbeit wurde ein System zur Bestimmung von personalisierten Vibrationsmustern entwickelt. 
F{\"u}r das System habe man die Komponenten f{\"u}r einen Evolution{\"a}ren Algorithmus entworfen und implementiert.
Die vier Phasen des Systems wurden mit 32 Probanden durchgef{\"u}hrt. 

Man darf auch nicht unerw{\"a}hnt lassen, dass die Probanden niemals eine R{\"u}ckmeldung erhalten haben, dass Sie das Signal richtig oder falsch bewertet hatten.
Die Zeit der Bewertung sowie die Genauigkeit stieg pro Iteration des Algorithmus an.
%Die Probanden haben niemals eine R{\"u}ckmeldung dar{\"u}ber erhalten, ob Sie richtig oder falsch liefen.
Im Anschluss des Algorithmus hat man die genetischen Vibrationssignale gegen{\"u}ber dem generischen Vibrationssignalen bewerten lassen. 
Die Ergebnisse zeigen, dass die personalisierten Vibrationssignale signifikant besser erkannt worden sind, als die generischen Vibrationssignale.
W{\"a}hrend des Algorithmus haben die vier Iterationen kaum Auswirkungen auf die Stimmung gehabt.
Die gr{\"o}{\ss}te Schwierigkeit, die die Probanden hatten, waren die Grenzen zwischen den Signaltypen zu definieren.
Somit hat man an den Grenzen ein vermehrtes Aufkommen der Wiederholungen eines Signals.

Man kann feststellen, das je mehr Signale man in ein Muster besitzt, desto h{\"o}her wird die Wahrscheinlichkeit, dass die Muster nicht korrekt erkannt werden. 
Bei einigen Probanden war unter der Betrachtung einer kleinen Abweichung der Mittelwert der bestimmten Anfangsgrenzen der personalisierte Wert.
Die Genauigkeit hat sich mit der Erh{\"o}hung der Komplexit{\"a}t f{\"u}r leicht zu merkende Muster nicht verschlechtert.

Da die Studie nur mit einer Anzahl von 32 Probanden durchgef{\"u}hrt wurde, kann man diese Ergebnisse der Gruppierungen nur mit Vorbehalt betrachten, da es diese Ergebnisse nicht einfach auf eine gr{\"o}{\ss}ere Anzahl an Personen schlie{\ss}en kann. 
Somit hei{\ss}t es nicht, dass Personen die weiblich sind immer schlechter Signale erkennen als m{\"a}nnliche Personen. 
Wenn man dies {\"u}berpr{\"u}fen wollen w{\"u}rde, m{\"u}sste man die Studie auf eine gr{\"o}{\ss}ere Anzahl an Probanden erweitern. 
Diese Werte sind nur die Ergebnisse, die aus meiner Studie dabei herausgekommen sind.

Anhand der in dieser Arbeit erkannten Erkenntnisse kann man f{\"u}r weitere Forschungen verwenden, indem man die Zeit der einzelnen Phasen weites gehend verringern w{\"u}rde. 
Mittels dem Evolution{\"a}ren Algorithmus muss man eine die komplette Population f{\"u}r eine Iteration bewerten lassen, dies hat zur Folge, dass man auf eine hohe Anzahl an Bewertungen gelangen kann und dementsprechend viel Zeit in Anspruch nehmen kann.
Als alternative kann man es mit einem schneller konvergierendes L{\"o}sungsverfahren probieren. 

Des Weiteren kann man anhand von optimierten Evolution{\"a}ren Algorithmen versucht werden eine schnellere Konvergenz der personalisierten Werte zu erhalten. 

Man sieht jedoch anhand dieser Arbeit, dass die Hypothese mit signifikanten Werten best{\"a}tigt wurde.

Ich bin der Meinung, dass man die personalisierten Vibrationen nicht au{\ss}er Acht lassen sollte, da man sieht, das selbst gro{\ss}e Hersteller noch keine optimalen L{\"o}sungen haben bieten. 



%%% Local Variables: 
%%% mode: latex
%%% TeX-master: "diplarb"
%%% End: 
