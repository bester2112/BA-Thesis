%% Zielsetzung der Arbeit.tex
%% $Id: Zielsetzung der Arbeit.tex 28 2007-01-18 16:31:32Z bless $
%%


Mit dieser Bachelorarbeit verfolge ich das Ziel personalisierte Vibrationssignale zu erstellen. Hierbei werden drei verschiedene Vibrationssignale f{\"u}r einen Nutzer so angepasst, dass die Werte speziell f{\"u}r den Benutzer bestimmt werden. Zur Bestimmung der eines Vibrationssignals wird ein Evolution{\"a}rer Algorithmus verwendet. Nachdem der Evolution{\"a}re Algorithmus passende Wert gefunden hat, wird {\"u}berpr{\"u}ft, wie gut die personalisierten Vibrationssignale im vergleich zu vorgegebenen Werten erkannt werden. 


//Die Aufgabe besteht darin, f{\"u}r ein Individuum mit einem Programm drei verschiedene Vibrationssignale f{\"u}r Ihn so anzupassen, dass die Werte von Ihm besser erkannt werden als vorgegebene Werte. Dabei wird zur Bestimmung eines Vibrationssignals ein Evolution{\"a}rer Algorithmus verwendet. Nachdem dieser passende Wert gefunden worden ist, wird {\"u}berpr{\"u}ft, wie gut im Vergleich zu Vorgegebenen Werten die Daten erkannt werden.

Somit wollte ich herausfinden, ob es m{\"o}glich ist f{\"u}r jedes Individuum eine eigene passende L{\"a}nge und St{\"a}rke von Vibrationssignalen zu erstellen, so dass die Kombination der Signale noch erkannt werden.

Die Hypothese die ich mit dieser Bachelorarbeit beantworten m{\"o}chte ist, ob man mittels personalisierten Vibrationen eine Folge von Vibrationen besser unterscheiden als vorgegebene Vibrationen.
Um dies beantworten zu k{\"o}nnen verwende ich Wearable zur Darstellung der Vibrationssignale und einen Genetischen Algorithmus um die personalisierten Vibrationen zu ermitteln.
