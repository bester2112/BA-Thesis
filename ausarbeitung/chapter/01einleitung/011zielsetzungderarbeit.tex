%% Zielsetzung der Arbeit.tex
%% $Id: Zielsetzung der Arbeit.tex 28 2007-01-18 16:31:32Z bless $
%%

Jedes Jahr werden neue Smartphones ver{\"o}ffentlicht und vorgegaukelt das alles neu entwickelt wurde. Dabei {\"a}ndern sich nur die inneren Komponenten durch einen neuen Prozessor und kleinen Softwareupdates sowie eine minimale Verbesserung an der Kamera. 
Es wird immer davon gesprochen, dass man dem Benutzer mit dem neuen Smartphone ein besseres Erlebnis bieten m{\"o}chte. 


Durch diese Abhandlung widmet sich die Frage, wie weit kann ein Smartphone individuell personalisiert werden. 
%Dabei stellt man die Frage, wie weit kann man ein Smartphone so personalisieren, dass es genau f{\"u}r eine Person angepasst ist? 
Unter den Standardeinstellungen kann man zwar ein Hintergrundbild, einzelne Klingelt{\"o}ne, Schriftgr{\"o}{\ss}en und die Platzierungen der Apps anpassen. 
Aber das ist alles nur f{\"u}r die visuellen sowie audiovisuellen Wahrnehmungen. 
Wie soll man unterschiedliche Informationen wahrnehmen, wenn das Smartphone auf dem Tisch oder in der Hosentasche ist und das Smartphone auf stumm gestellt ist? 

Aus diesem Grund verfolge man das Ziel, personalisierte Vibrationssignale mittels dem vorgegebenen Wearable zu erstellen.
%Im Rahmen dieser Bachelorarbeit verfolge man das Ziel, personalisierte Vibrationssignale zu erstellen.
Hierbei werden drei verschiedene Vibrationssignale f{\"u}r einen Nutzer so angepasst, dass die Werte speziell f{\"u}r den Benutzer bestimmt werden. 
Zur Bestimmung der Vibrationssignale wird ein Evolution{\"a}rer Algorithmus verwendet. Nachdem diese Handlungsvorschrift die passenden Werte gefunden hat, wird {\"u}berpr{\"u}ft, wie gut die Benutzer die personalisierten Vibrationssignale im Vergleich zu vorgegebenen Werten erkennen k{\"o}nnen.
%wie gut die personalisierten Vibrationssignale im Vergleich zu vorgegebenen Werten erkannt werden. 
Es hat sich dabei die Hypothese gebildet, dass die personalisierten Vibrationen besser erkannt werden als die vorgegebenen Vibrationen.

%Die Aufgabe besteht darin, f{\"u}r ein Individuum mit einem Programm drei verschiedene Vibrationssignale f{\"u}r Ihn so anzupassen, dass die Werte von Ihm besser erkannt werden als vorgegebene Werte. Dabei wird zur Bestimmung eines Vibrationssignals ein Evolution{\"a}rer Algorithmus verwendet. Nachdem dieser passende Wert gefunden worden ist, wird {\"u}berpr{\"u}ft, wie gut im Vergleich zu vorgegebenen Werten die Daten erkannt werden.

%Somit wollte man herausfinden, ob es m{\"o}glich ist f{\"u}r jedes Individuum eine eigene passende L{\"a}nge und St{\"a}rke von Vibrationssignalen zu erstellen, sodass die Kombination der Signale noch erkannt werden.

%Die Hypothese, die man mit dieser Bachelorarbeit beantworten m{\"o}chte ist, ob man mittels personalisierten Vibrationen eine Folge von Vibrationen besser unterscheiden als vorgegebene Vibrationen.
%Um dies beantworten zu k{\"o}nnen verwende man ein Wearable zur Darstellung der Vibrationssignale und den Evolution{\"a}ren Algorithmus um die personalisierten Vibrationen zu ermitteln.
