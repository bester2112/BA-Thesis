%% Studiendesign.tex
%% $Id: Studiendesign.tex 4 2005-10-10 20:51:21Z bless $
%%

\chapter{Studiendesign}
\label{ch:Studiendesign}

%% ==============================
\section{Studiendesign}
%% ==============================
\label{ch:Evolutionärer Algorithmus:sec:Studiendesign}
TODO Benutzerflussdiagramm

Im folgenden Benutzerflussdiagramm hat man sich orientiert um die Studie zu entwerfen. (Sample of convenience) Über Mailinglisten und Bekanntenkreis haben sich 32 Probanden bereiterklärt an der Studie teilzunehmen. Dabei waren 72 Prozent Männer und 28 Prozent Frauen. Das Alter der Probanden war von 12 bis 54 Jahren vertreten und das Durchschnittsalter war 22 Jahre. Die Studie hat zwischen 30 Minuten und einer Stunde gedauert.
Die Studien wurden an drei verschiedenen Orten durchgeführt, im TECO in Karlsruhe, in einem Seminarraum an der Hochschule Darmstadt und in einem Arbeitszimmer in Meschede.

Für jeden teilgenommenen Probanten ist der gleiche Ablauf durchgeführt. 
Vor der Studie wurde ein Termin mit dem Probanten vereinbart. Nachdem der Probant zur abgemachten Zeit am vorgegebenen Ort angekommen ist, wurde Ihm erklärt wofür die Studie ist, was man mit der Studie herausfinden will und welche Erwartungen man an den Probanten hat. 
Nachdem der Probant alles verstanden hat und die Einverständniserklärung verstanden und unterschirieben hat, wurde Ihm das Armband angezogen. 

Für die Studie hat man ein Programm mit einer Grafischen Oberflaeche (GUI) entworfen, mit der es möglich war die ganze Studie durchzufueren. 
Dabei wurden dem Probanden ein paar Personalien abgefragt, wie das Alter, das Geschlecht, ob sich die Person als Musikalisch empfindet, ob man Computerspiele spielen würde, ob man schon einmal eine Smartwatch benutzt habe und ob die Person schon mal ein Tactiles Geraet benutzt habe. Falls vom Probanden Fragen waehrend der Studie Fragen aufgekommen sind wurden diese sofort beantwortet. 

Nach der Aufnahme der Personalien, wurde dem Benutzer erklaert, was Ihn als naechstes erwartet und von Ihmverlangt wird. 
Man hat dem Nutzer im ersten Schritt 10 Signale abgespielt, um Ihn ein Gefuehl fuer Signale zu geben. Im Anschluss wurde dem Probanten jedes Signal erneut einmalig abgespielt. Dabei sollte er das Signal zu drei jeweiligen Kategorien zuordnen. Diese Kategorien waren ob es ein Kurzes, Mittleres oder Langes Signal fuer Ihn gewesen ist. Dieser Schritt war dafuer notwendig um fuer den Benutzer die Grenzen fuer die jeweilige Kategorien Kurz, Mittel und Lang zu bestimmen. 
Diese Grenzen sind fuer die Initialisierung des Algorithmus notwendig gewesen. 

Als die 10 Signale bewertet wurden, wurde der Benutzer darüber aufgeklärt, was Ihm als nächsten Schritt erwartet. Es wurde Ihm ein anhand seiner Eingaben ein zufaelliges Signal abgespielt, dass er bewerten sollte (BILD). Anhand der drei Fragen wurde das Signal bewertet. Um eine Itteration komplett zu bewerten wurde dieser Vorgang 30 mal wiederholt. Im Anschluss wurde gefragt wie der Benutzer sich derzeit fuehlt (BILD). Anhand einer komplett bewerteten Itteration wurde dem Benutzer neue Werte berechnet. Es wurden insgesamt vier Itterationen durchgefuehrt um einen möglichst genauen Wert füer den Benutzer zu bestimmen.

Im letzten Schritt wurde dem Benutzer aufgeklaert, dass ab dem Zeitpunkt nur noch Folgen von Signalen, die man ab jetzt Muster nennt, abgespielt werden. Die Probanden sollten angeben in welcher Reihenfolge was fuer Signaltypen abgespielt wurden. Es wurde für alle Probanden im Vorfeld alle Muster definiert, damit jeder die gleichen Muster abgespielt bekommt. Es gab zwei Arten von Muster, die generischen Muster und die genetischen Muster. Der einzige Unterschied zwischen den beiden Arten waren die Werte, die die Signale in einem Muster zugewiesen wurden. Das bedeutet es wurden zwei mal das selbe Muster abgespielt mit lediglich anderen Werten. Die Genetischen Muster hatten die Werte, die nach dem Algorithmus erzeugt wurden übernommen, wobei die generischen Muster einen vordefinierten Wert zugewiesen bekommen hat. Der generische Wert ist fuer jeden Probanden gleich gewesen.
Dabei gab es Muster mit drei, vier und fuenf Signalen. Nacheinander wurde dem Nutzer zuerst alle Muster mit drei Signalen. Dabei wurde das genetische Muster abwechselnd zum generischen Muster abgespielt. 

Nachdem alle Muster von dem Probanden bewertet wurden, haben Sie Ihre e-Mail noch angegeben um an einer automatischen Verlosung von zwei Gutscheinen teilzunehmen. Bei Interesse wurde Ihnen Ihre Werte gezeigt und erklärt, was genau im Hintergrund passiert worden ist. Während der ganzen Studie standen dem Probanten ausreichend Suesigkeiten zur Verfuegung, bei denen Sie sich frei bedienen konnten.