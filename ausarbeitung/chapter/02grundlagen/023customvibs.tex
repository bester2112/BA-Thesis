%% custom vibs.tex
%% $Id: Genetischer Algorithmus.tex 28 2007-01-18 16:31:32Z bless $
%%

Da heutzutage beinahe jedes Ger�t ein Vibtarionsmotor verbaut hat, sei es das Handy, die Smartwatches oder Fitnessarmb�nder (uvm.), werde ich im folgenden noch auf einige aktuelle Technologien und deren Umsetzung der personalisierten Vibtationsmuster zu sprechen kommen. 

%% ==============================
\subsection{Taptic Engine}
%% ==============================
\label{ch:Grundlagen:sec:RelatedWork:subsec:TapticEngine}

Die Taptic Engine ist eine von der Firma Apple eine eigen entwickeltes Vibrationsmotor, dass heutzutage in nahezu allen Apple Produkten vertreten ist. Das erste Ger�t was die Taptic Engine bekommen hatte, war die Apple Watch. Der Name bildet sich aus dem Wort "Taktil" und "Haptisch". 
Trotz des neu erfinden einer mechanischen R�ckmeldung bietet Apple keine Personalisierung, wie lange eine R�ckmeldung erfolgen soll, f�r die Apple Watch an. Die Einstellungsm�glichkeiten an der Apple Watch ist lediglich die St�rke der Vibration. Diese ist in 3 Stärkestufen unterteilt. Dabei kann man aber nicht wirklich von einer Personalisierung sprechen. 

(BILD) 
 
%% ==============================
\subsection{Personalisierte Vibration}
%% ==============================
\label{ch:Grundlagen:sec:RelatedWork:subsec:PersonalisierteVibration}

(BILD)

Der Hersteller Apple hat auch bei dem iPhone eine M�glichkeit geboten eigene Vibrationsmuster zu erstellen, jedoch mit Einschr�nkungen.
Wenn man in die jeweilige Einstellung der iPhones gelangt, erscheint das folgende Bild. Beim dr�cken auf das Display wird an der Stelle eine Vibration erzeugt. Man hat 10 Sekunden um ein eigenes Muster zu erzeugen, indem man wiederholt auf den Bildschirm dr�ckt. An der Stelle, an der man den Bildschirm ber�hrt hat, erscheint visuell um der Position ein Kreis. Die erzeugten Vibrationen werden in einer Leiste visuell angezeigt. 
Man kann sich beliebig viele Vibrationsmuster speichern, die bis zu 10 Sekunden lang sind. \cite{fleizach2016custom}

Die Einschr�nkung die man hier nennen muss ist, dass man die Vibrationsmuster nur f�r Systeminterne Funktionen benutzen kann. Dies bedeutet, dass man die Funktionen f�r Klingelt�ne, Nachrichtent�ne, Erinnerungshinweise, Kalenderhinweise (o.�.) hinzuf�gen kann. F�r eine andere Anwendung, die nicht im Betriebsystem integriert ist, ist das nicht m�glich. Somit k�nnen Benachrichtigungen von anderen Entwicklern keine eigene Vibrationsmuster zuweisen. Somit kann man bei Benachrichtigungen von anderen Entwicklern nicht anhand der Vibrationen des iPhones  unterscheiden.

%% ==============================
\subsection{Personalisierte Smartwatch}
%% ==============================
\label{ch:Grundlagen:sec:RelatedWork:subsec:PersonalisierteSmartwatch}

Das Ger�t, dass es nach meiner Recherche am besten gel�st hat, ist eine Smartwatch von einem kleinen StartUp namens Martian. 
Das Startup hat eine Uhr hergestellt, mit der man mittels einer App auf dem Smartphone die Vibrationsmuster selbst anpassen kann. 
Die App unterst�tzt eine gro�e Anzahl an Applications, von anderen Herstellern, die Benachrichtigungen senden. Ein Vibrationsmuster f�r die Uhr kann man aus mit zu 4 Signalen auf der Uhr darstellen lassen. Die Signale sind als Lang, Kurz und Pause festgelegt. 
Somit kann man mittels der Vibration der Uhr herausfinden, welche App gerade eine Benachrichtigung auf mein Handy gesendet hat. Die L�nge und St�rke eines Signals ist schon im Vorfeld festgelegt.

%% ==============================
\subsection{Fazit}
%% ==============================
\label{ch:Grundlagen:sec:RelatedWork:subsec:PersonalisierteSmartwatch}
Bei sehr vielen Herstellern ist es aktuell noch gar nicht m�glich eigene Vibrationsmuster zu erstellen. Bei Android Ger�ten ist es aktuell so, dass man nur aus einer Menge von wenigen Vordefinierten Vibrationsmustern sich nur eines ausw�hlen kann. Einige haben dieses Problem erkannt und eine eigene App entwickelt und in den Store gestellt.

(BILD)
