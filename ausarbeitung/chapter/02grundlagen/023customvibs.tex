%% custom vibs.tex
%% $Id: Genetischer Algorithmus.tex 28 2007-01-18 16:31:32Z bless $
%%

Da heutzutage beinahe jedes Ger�t ein Vibtarionsmotor verbaut hat, sei es das Handy, die Smartwatches oder Fitnessarmb�nder (uvm.), werde ich im folgenden noch auf einige aktuelle Technologien und deren Umsetzung der personalisierten Vibtationsmuster zu sprechen kommen. 

%% ==============================
\subsection{Taptic Engine}
%% ==============================
\label{ch:Grundlagen:sec:RelatedWork:subsec:TapticEngine}

Die Taptic Engine ist eine von der Firma Apple eine eigen entwickeltes Vibrationsmotor, dass heutzutage in nahezu allen Apple Produkten vertreten ist. Das erste Ger�t was die Taptic Engine bekommen hatte, war die Apple Watch. Der Name bildet sich aus dem Wort "Taktil" und "Haptisch". 
Trotz des neu erfinden einer mechanischen R�ckmeldung bietet Apple keine Personalisierung, wie lange eine R�ckmeldung erfolgen soll, f�r die Apple Watch an. Die Einstellungsm�glichkeiten an der Apple Watch ist lediglich die St�rke der Vibration. Diese ist in 3 Stärkestufen unterteilt. Dabei kann man aber nicht wirklich von einer Personalisierung sprechen. 

(BILD) 

Aktuell ist es so, dass man bei dem Hersteller Samsung keinerlei Möglichkeit hat sich dort die Vibrationen zu Personalisieren. Man kann hier lediglich unter einer Handvoll vordefinierten Vibrationsmuster entscheiden. 
 
%% ==============================
\subsection{Personalisierte Vibration}
%% ==============================
\label{ch:Grundlagen:sec:RelatedWork:subsec:PersonalisierteVibration}

(BILD)

Der Hersteller Apple hat auch bei dem iPhone eine M�glichkeit geboten eigene Vibrationsmuster zu erstellen, jedoch mit Einschr�nkungen.
Wenn man in die jeweilige Einstellung der iPhones gelangt, erscheint das folgende Bild. Beim dr�cken auf das Display wird an der Stelle eine Vibration erzeugt. Man hat 10 Sekunden um ein eigenes Muster zu erzeugen, indem man wiederholt auf den Bildschirm dr�ckt. An der Stelle, an der man den Bildschirm ber�hrt hat, erscheint visuell um der Position ein Kreis. Die erzeugten Vibrationen werden in einer Leiste visuell angezeigt. 
Man kann sich beliebig viele Vibrationsmuster speichern, die bis zu 10 Sekunden lang sind. \cite{fleizach2016custom}

Die Einschr�nkung die man hier nennen muss ist, dass man die Vibrationsmuster nur f�r Systeminterne Funktionen benutzen kann. Dies bedeutet, dass man die Funktionen f�r Klingelt�ne, Nachrichtent�ne, Erinnerungshinweise, Kalenderhinweise (o.�.) hinzuf�gen kann. F�r eine andere Anwendung, die nicht im Betriebsystem integriert ist, ist das nicht m�glich. Somit k�nnen Benachrichtigungen von anderen Entwicklern keine eigene Vibrationsmuster zuweisen. Somit kann man bei Benachrichtigungen von anderen Entwicklern nicht anhand der Vibrationen des iPhones  unterscheiden.

%% ==============================
\subsection{Personalisierte Smartwatch}
%% ==============================
\label{ch:Grundlagen:sec:RelatedWork:subsec:PersonalisierteSmartwatch}

Das letzte Ger�t ist eine Smartwatch von einem kleinen StartUp namens Martian. 

Das letzte Gerät was ich ansprechen möchte, ist von keiner großen Firma wie Apple oder Samsung sondern ein StartUp namens Martian. Diese Firma hat eine Uhr hergestellt, die man mittels einer App auf dem Handy anpassen kann. Mittels der Application kann man sich für mehrere Hunderte Apps die Benachrichtigungen senden, ein Muster mit bis zu 4 Signalen auf der Uhr als Vibration darstellen lassen. Die Signale sind zwischen Lang, Kurz und Pause (also kein Signal) Darstellbar. 
Die Uhr hat mich als einziges Produkt überzeugt, dass man sich persönliche Vibrationsmuster erstellen und abspielen lassen kann. Natürlich ist die Länge und Stärke der Vibration damit vorgegeben.

Somit wollte ich herausfinden, ob es möglich ist für jedes Individuum eine eigene passende Länge und Stärke von Vibrationssignalen zu erstellen, so dass die Kombination der Signale noch erkannt werden.

Die Hypothese die ich mit dieser Bachelorarbeit beantworten möchte ist, ob man mittels personalisierten Vibrationen eine Folge von Vibrationen besser unterscheiden als vorgegebene Vibrationen.
Um dies beantworten zu können verwende ich Wearable zur Darstellung der Vibrationssignale und einen Genetischen Algorithmus um die personalisierten Vibrationen zu ermitteln.
