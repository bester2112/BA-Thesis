%% taktile geräate.tex
%% $Id: taktilegeraete.tex 28 2007-01-18 16:31:32Z bless $
%%

Ein Taktiles Ger{\"a}t ist ein Ger{\"a}t, dass Informationen an einen Menschen mitteilen m{\"o}chte, dies geschieht durch die Wahrnehmung der Haut. 
Ein Taktiles Ger{\"a}t ist ein Ger{\"a}t, dass Informationen durch die Wahrnehmung des Menschen mittels K{\"o}rperkontakt darstellet.  \cite{gemperle2001design}

Tatile Ger{\"a}te werden heutzutage {\"o}fter verwendet, als man es eigentlich wahrnimmt. 
Ein einfaches Beispiel ist das Handy. 
Eine Person hat sein eigenes Handy in der Hosentasche. Bei einer eingehenden Nachricht, muss der Benutzer mitgeteilt werden, dass eine Nachricht empfangen wurde.
Normalerweise geschieht das beim abspielen des Klingeltons. Falls man besch{\"a}ftigt ist und nicht durch ein lautes klingeln gest{\"o}rt werden will,  stellt man den Ton ab. 
Um dennoch darauf Aufmerksam zu machen, dass eine Nachricht eingetroffen ist, nutzt man die Vibration des Handys.

%Das Handy ist nur eins von vielen Beispielen, was man {\"u}ber den Alltag noch f{\"u}r Taktile Ger{\"a}te verwendet.
Ein weiteres Beispiel in der Taktile Displays benutzt werden ist f�r Blinde Menschen um wissen zu �bermitteln \cite{parente2003bats}. 
F�r Blinde gibt es spezielle Karten, die Konturen f�r die Umrisse der L�nder darstellen, solche Karten sind selten vertreten.
Um ein solches Ungleichgewicht zwischen den blinden und normalsehenden Sch�lern zu mindern, hat man sich ein System f�r entwickelt, das audio und taktile Eindr�cke vereint und somit 
Dabei hat man sich f�r einen r�umlich akustisches Soundsystem entschieden um den Sch�ler anhand Ger�uschen aus verschiedenen Richtungen ein Gef�hl zu geben soll wo man sich aktuell befindet und was um einen existiert. Dabei wurden nicht nur Tierger�usche und Verkehrsger�usche verwendet, sondern auch den Namen der Region, in der man sich aktuell befindet. 
Als Traktiles Device hat man sich an schon existierenden M�usen und Controllern bedient, die Force-Feedback besa�en. 

Die Karten die man verwendet hat sind Karten, diese sind nicht so detailliert wie normale Karten sind. Im Vergleich zu der normalen Karten hat man hier weniger Informationen die man �bertragen muss und die Details der L�ndergrenzen konnte man leichter darstellen. 

Damit ein Sch�ler eine Karte erkunden kann bewegt er mithilfe eines Eingabeger�ts �ber die Karte und dr�ckt eine Taste auf der Tastatur um Informationen �ber das Gebiet abzurufen, die �ber Ausdosignale �bermittelt werden. Das Taktile Device wurde verwendet um Grenz�berg�nge zu mittels Vibrationen zu signalisieren.


