%% taktile geräate.tex
%% $Id: taktilegeraete.tex 28 2007-01-18 16:31:32Z bless $
%%

Ein Taktiles Ger{\"a}t ist ein Ger{\"a}t, dass Informationen an einen Menschen mitteilen m{\"o}chte, dies geschieht durch die Wahrnehmung der Haut. 
Ein Taktiles Ger{\"a}t ist ein Ger{\"a}t, dass Informationen durch die Wahrnehmung des Menschen mittels K{\"o}rperkontakt darstellet.  \cite{gemperle2001design}

Tatile Ger{\"a}te werden heutzutage {\"o}fter verwendet, als man es eigentlich wahrnimmt. 
Ein einfaches Beispiel ist das Handy. 
Eine Person hat sein eigenes Handy in der Hosentasche. Bei einer eingehenden Nachricht, muss der Benutzer mitgeteilt werden, dass eine Nachricht empfangen wurde.
Dies geschieht normalerweise mit dem Klingelton. Wenn man jetzt besch{\"a}ftigt ist und nicht durch ein lautes klingeln gest{\"o}rt werden m{\"o}chte, dann stellt man den Ton ab. 
Um dennoch darauf Aufmerksam zu machen, dass eine Nachricht eingetroffen ist, nutzt man die Vibration des Handys.

Das Handy ist nur eins von vielen Beispielen, was man {\"u}ber den Alltag noch f{\"u}r Taktile Ger{\"a}te verwendet.

//Man nutzt Vibrationen um darauf aufmerksam zu machen, dass trotz abgeschaltetem Klingelton, eine Nachricht empfangen wurde. 

//Falls eine Person das Handy in der Hosentasche haben sollte und eine eingehende Nachricht empfangen wurde, will das Handy dem Benutzer das mitteilen.
Dies passiert im Normalfall durch einen Klingelton. Um als Taktiles Ger{\"a}t definiert zu werden, muss es dem Benutzer per K{\"o}rperkontakt mitteilen, dass eine Nachricht eingegangen ist. 
Das geschieht nicht durch den Klingelton sondern durch Vibration. 