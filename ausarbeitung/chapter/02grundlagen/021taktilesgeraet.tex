%% taktile ger{\"a}ate.tex
%% $Id: taktilegeraete.tex 28 2007-01-18 16:31:32Z bless $
%%

Ein Taktiles Ger{\"a}t ist ein Ger{\"a}t, dass Informationen an einen Menschen mitteilen m{\"o}chte, dies geschieht durch die Wahrnehmung der Haut. \cite{gemperle2001design}

Taktile Ger{\"a}te werden heutzutage {\"o}fter verwendet, als man es eigentlich wahrnimmt. 
Ein einfaches Beispiel ist das Handy. 
Eine Person hat sein eigenes Handy in der Hosentasche. Bei einer eingehenden Nachricht, muss der Benutzer mitgeteilt werden, dass eine Nachricht empfangen wurde.
Normalerweise geschieht das beim Abspielen des Klingeltons. Falls man besch{\"a}ftigt ist und nicht durch ein lautes Klingeln gest{\"o}rt werden will, stellt man den Ton ab. 
Um dennoch darauf Aufmerksam zu machen, dass eine Nachricht eingetroffen ist, nutzt man die Vibration des Handys.

%Das Handy ist nur eins von vielen Beispielen, was man {\"u}ber den Alltag noch f{\"u}r Taktile Ger{\"a}te verwendet.
Ein weiteres Beispiel in der Taktile Displays benutzt werden ist f{\"u}r Blinde Menschen, um Wissen zu {\"u}bermitteln \cite{parente2003bats}. 
F{\"u}r Blinde gibt es spezielle Karten, die Konturen f{\"u}r die Umrisse der L{\"a}nder darstellen, solche Karten sind selten vertreten.
Um ein solches Ungleichgewicht zwischen den blinden und normal sehenden Sch{\"u}lern zu mindern, hat man sich ein System entwickelt, dass audiovisuelle und taktile Eindr{\"u}cke vereint. 
Dabei hat man sich f{\"u}r eines r{\"a}umlich akustisches Soundsystem entschieden, um den Sch{\"u}ler anhand Ger{\"a}uschen aus verschiedenen Richtungen ein Gef{\"u}hl zu geben, an welchen Orten man sich aktuell befindet und was um einen herum passiert. 
Dabei wurden nicht nur Tierger{\"a}usche und Verkehrsger{\"a}usche verwendet, sondern auch den Namen der Region, in der man sich aktuell befindet abgespielt. 
Als Taktiles Device hat man sich an schon existierenden M{\"a}usen und Controllern bedient, die Force-Feedback besa{\ss}en. 

Die verwendeten Karten waren nicht so detailliert wie in der Realit{\"a}t.
%Die Karten, die man verwendet hat, sind nicht so detailliert, wie normale Karten. 
Im Vergleich zu der normalen Karten hat man hier weniger Informationen, die man {\"u}bertragen muss und die Details der L{\"a}ndergrenzen konnte man leichter darstellen. 

Damit ein Sch{\"u}ler eine Karte erkunden konnte, bewegt er sich mithilfe eines Eingabeger{\"a}ts {\"u}ber die Karte und dr{\"u}ckt eine Taste auf der Tastatur um Informationen {\"u}ber das Gebiet abzurufen, diese wurden {\"u}ber Audiosignale {\"u}bermittelt. Das Taktile Device wurde verwendet um Grenz{\"u}berg{\"a}nge zu mittels Vibrationen zu signalisieren.


