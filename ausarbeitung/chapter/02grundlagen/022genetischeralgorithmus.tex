%% Genetischer Algorithmus.tex
%% $Id: Genetischer Algorithmus.tex 28 2007-01-18 16:31:32Z bless $
%%

Die Evalution{\"a}ren Algorithmen sind stochastische Optimierungsverfahren. Man findet damit nicht die beste L{\"o}sung f{\"u}r ein Problem, jedoch findet man eine gute Ann{\"a}herung.
Dabei hat man sich bei den Evolution{\"a}ren Algorithmen an der Biologischen Evolution von Darwin inspirieren lassen. \cite{selzam2003genetische}

In der Biologie ist jeder lebende Organismus ein Induvidiuum.
Jedes Induvidiuum besitzt Erbinformationen in der Form von Chromosomen. Die Erbinformationen werden auch \textbf{Gene} oder \textbf{DNA} genannt.
Eine Gruppe von Induviduen wird als \textbf{Population} bezeichnet.

Der Algorithmus wurde anhand der folgenden Merkmale erstellt.

%% ==============================
\subsection{Selektion}
%% ==============================
\label{ch:Grundlagen:sec:Taktile Ger{\"a}te:subsec:Selektion}

Eine \textbf{Selektion} ist ein Mechanismus, bei dem man sich zwei Individuen ausw{\"a}hlt, um anschlie{\ss}end die Gene der beiden Individuen, mittels der sogenannten \textbf{Rekombination}, zu kombinieren. Dabei gibt es verschiedene Selektionsstrategien. Man versucht die Individuen zu finden, um eine bestm{\"o}glichste L{\"a}sung f{\"u}r ein Problem zu liefern.

%% ==============================
\subsection{Variation}
%% ==============================
\label{ch:Grundlagen:sec:Taktile Geräte:subsec:Variation}

Man sollte eine zahlreiche Variation von Genen der einzelnen Individuen besitzen. 
Denn nehme anhand einem Beispiel von S{\"u}{\ss}igkeiten einmal kurz an, dass alle S{\"u}{\"ss}igkeiten die gleiche Farbe und die gleiche Form haben. Wenn man sich jetzt zwei zuf{\"a}llige S{\"u}{\ss}igkeiten ausw{\"a}hlen w{\"u}rde, h{\"a}tten Sie keinerlei unterschiede und so w{\"u}rde das Nachkommen die gleichen Gene besitzen. Damit genau das nicht auftaucht, versucht man zu Beginn eine gro{\ss}e Variation an Individuen zu erzeugen und diese als Anfangs Population f{\"u}r einen Evolution{\"a}ren Algorithmus zu nutzen. Mittels Rekombination und Mutation wird dabei ein neues Induvidiuum f{\"u}r die n{\"a}chste Generation erzeugt. 


//Durch Rekombination und Mutation der DNA der Induviduuen erh{\"a}lt man eine Variation. 
//Die Rekombination und Mutation erzeugt lediglich ein neues Induviduum aus der DNA der zuvor selektierten Induviduen.

Um die n{\"a}chsten Generationen zu bilden wird eine Rekombination und Mutation auf die DNA der Selektierten Induviduen ausgef{\"u}hrt. 
Die Variation der DNA  spielt eine wichtige Rolle, denn die ist f{\"u}r die nachfolgende Rekombination und Mutation entscheident f{\"u}r die n{\"a}chsten Generationen. 
Um bei der Selektion unterschiedliche Induviduuen ausgew{\"a}hlt werden k{\"o}nnen, ben{\"o}tigt man zun{\"a}chst eine Variation 

%% ==============================
\subsection{Vererbung / Gendrift}
%% ==============================
\label{ch:Grundlagen:sec:Taktile Ger{\"a}te:subsec:Variation}
Bla fasel \dots

%% ==============================
\subsection{Allgeimeiner Vorgang eines Evolution{\"a}ren Algorithmus}
%% ==============================
\label{ch:Grundlagen:sec:Taktile Ger{\"a}te:sec:Allgeimeiner Vorgang eines Evolution{\"a}ren Algorithmus}
Ein Evolution{\"a}rer Algorithmus besitzt grunds{\"a}tzlich immer die gleichen Komponenten die miteinander.

%% ==============================
\subsection{Initialisierung}
%% ==============================
\label{ch:Grundlagen:sec:Initialisierung}
Man erzeugt sich zur Initialisierung eine Population von Individuen, die eine Variation von Genen besitzt. 

%% ==============================
\subsection{Bewertung der Individuen}
%% ==============================
\label{ch:Grundlagen:sec:Bewertung der Individuen}
Bevor man eine neue Population f{\"u}r die n{\"a}chste Generation berechnen kann, muss man zuerst mithilfe einer Fitnessfunktion jedes Induvidiuum einen Fitnesswert bestimmen. 

%% ==============================
\subsection{Selektion}
%% ==============================
\label{ch:Grundlagen:sec:Selektion}
Anhand dem Fitnesswert der Induviduen werden zwei zuf{\"a}llige Induviduen bestimmt. Diese selektierten Induviduen werden als Eltern f{\"u}r die n{\"a}chste Generation benutzt.
Bei der Selektion wird ein hoher Fitnesswert bevorzugt.

%% ==============================
\subsection{Rekombination}
%% ==============================
\label{ch:Grundlagen:sec:Rekombination}
Die Gene der ausgew{\"a}hlten Eltern werden miteinander kombiniert und bilden die Gene des neuen Induvidiuum f{\"u}r die Population der n{\"a}chsten Generation. Hier gibt es verschiedene Kombinationsm{\"o}glichkeiten, die angewendet werden k{\"o}nnen.

%% ==============================
\subsection{Mutation}
%% ==============================
\label{ch:Grundlagen:sec:Mutation}
Bei dem erzeugen Induvidiuum besteht eine Chance, dass die kombinierten Gene durch eine Mutation ver{\"a}ndert werden. 

%% ==============================
\subsection{Wiederholung durch neuer Generation}
%% ==============================
\label{ch:Grundlagen:sec:Wiederholung durch neuer Generation}
Der Vorgang der Selektion, Rekombination und Mutation wird so oft ausgef{\"u}hrt, bis man eine neue Population hat, die genauso gro{\ss} ist, wie die Anfangapopulation.
Nachdem die neue Population erzeugt wurde, wird diese durch die Anfangspopulation ersetzt und man f{\"u}hrt den Algorithmus erneut aus. 
Dies geschieht so lange, bis man eine hinreichende L{\"o}sung für das Problem hat. 




Ich habe meinen Evolution{\"a}ren Algorithmus so angepasst, dass bei mir ein Induviduum ein Signal ist. 
Ich habe dem Benutzer das Signal mit dem Wearable abspielen lassen und im Anschluss Fragen beantworten lassen. 
Er sollte bewerten wie gut er das Signal erkannt hat. Die Bewertung vom Benutzer war entscheidend um nach der kompletten Bewertung der Population 