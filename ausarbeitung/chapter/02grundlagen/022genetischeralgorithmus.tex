%% Genetischer Algorithmus.tex
%% $Id: Genetischer Algorithmus.tex 28 2007-01-18 16:31:32Z bless $
%%

Die Evalutionären Algorithmen sind stochastische Optimierungsverfahren. Man findet damit nicht die beste Lösung für ein Problem, jedoch findet man eine gute Annäherung.
Dabei hat man sich bei den Evolutionären Algorithmen an der Biologischen Evolution von Darwin inspirieren lassen.

In der Biologie ist jeder lebende Organismus ein Induvidiuum.
Jedes Induvidiuum besitzt Erbinformationen in der Form von Chromosomen. Die Erbinformationen werden auch \textbf{Gene} oder \textbf{DNA} genannt.
Eine Gruppe von Induviduen wird als \textbf{Population} bezeichnet.

Der Algorithmus wurde anhand der folgenden Merkmale erstellt.

%% ==============================
\section{Selektion}
%% ==============================
\label{ch:Grundlagen:sec:Taktile Geräte:sec:Selektion}

Eine \textbf{Selektion} ist ein Mechanismus, bei dem man sich zwei Individuen auswählt, um anschließend die Gene der beiden Individuen, mittels der sogenannten \textbf{Rekombination}, zu kombinieren. Dabei gibt es verschiedene Selektionsstrategien. Man versucht die Individuen zu finden, um eine bestmöglichste Lösung für ein Problem zu liefern.

%% ==============================
\section{Variation}
%% ==============================
\label{ch:Grundlagen:sec:Taktile Geräte:sec:Variation}

Man sollte eine zahlreiche Variation von Genen der einzelnen Individuen besitzen. 
Denn nehme anhand einem Beispiel von Süßigkeiten einmal kurz an, dass alle Süßigkeiten die gleiche Farbe und die gleiche Form haben. Wenn man sich jetzt zwei zufällige Süßigkeiten auswählen würde, hätten Sie keinerlei unterschiede und so würde das Nachkommen die gleichen Gene besitzen. Damit genau das nicht auftaucht, versucht man zu Beginn eine große Variation an Individuen zu erzeugen und diese als Anfangs Population für einen Evolutionären Algorithmus zu nutzen. Mittels Rekombination und Mutation wird dabei ein neues Induvidiuum für die nächste Generation erzeugt. 


//Durch Rekombination und Mutation der DNA der Induviduuen erhält man eine Variation. 
//Die Rekombination und Mutation erzeugt lediglich ein neues Induviduum aus der DNA der zuvor selektierten Induviduen.

Um die nächsten Generationen zu bilden wird eine Rekombination und Mutation auf die DNA der Selektierten Induviduen ausgeführt. 
Die Variation der DNA  spielt eine wichtige Rolle, denn die ist für die nachfolgende Rekombination und Mutation entscheident für die nächsten Generationen. 
Um bei der Selektion unterschiedliche Induviduuen ausgewählt werden können, benötigt man zunächst eine Variation 

%% ==============================
\section{Vererbung / Gendrift}
%% ==============================
\label{ch:Grundlagen:sec:Taktile Geräte:sec:Variation}
Bla fasel \dots

%% ==============================
\section{Allgeimeiner Vorgang eines Evolutionären Algorithmus}
%% ==============================
\label{ch:Grundlagen:sec:Taktile Geräte:sec:Allgeimeiner Vorgang eines Evolutionären Algorithmus}
Ein Evolutionärer Algorithmus besitzt grundsätzlich immer die gleichen Komponenten die miteinander.

%% ==============================
\section{Initialisierung}
%% ==============================
\label{ch:Grundlagen:sec:Initialisierung}
Man erzeugt sich zur Initialisierung eine Population von Individuen, die eine Variation von Genen besitzt. 

%% ==============================
\section{Bewertung der Individuen}
%% ==============================
\label{ch:Grundlagen:sec:Bewertung der Individuen}
Bevor man eine neue Population für die nächste Generation berechnen kann, muss man zuerst mithilfe einer Fitnessfunktion jedes Induvidiuum einen Fitnesswert bestimmen. 

%% ==============================
\section{Selektion}
%% ==============================
\label{ch:Grundlagen:sec:Selektion}
Anhand dem Fitnesswert der Induviduen werden zwei zufällige Induviduen bestimmt. Diese selektierten Induviduen werden als Eltern für die Nächste Generation benutzt.
Bei der Selektion wird ein hoher Fitnesswert bevorzugt.

%% ==============================
\section{Rekombination}
%% ==============================
\label{ch:Grundlagen:sec:Rekombination}
Die Gene der ausgewählten Eltern werden miteinander kombiniert und bilden die Gene des neuen Induvidiuum für die Population der nächsten Generation. Hier gibt es verschiedene Kombinationsmöglichkeiten, die angewendet werden können.

%% ==============================
\section{Mutation}
%% ==============================
\label{ch:Grundlagen:sec:Mutation}
Bei dem erzeugen Induvidiuum besteht eine Chance, dass die kombinierten Gene durch eine Mutation verändert werden. 

%% ==============================
\section{Wiederholung durch neuer Generation}
%% ==============================
\label{ch:Grundlagen:sec:Wiederholung durch neuer Generation}
Der Vorgang der Selektion, Rekombination und Mutation wird so oft ausgeführt, bis man eine neue Population hat, die genauso groß ist, wie die Anfangapopulation.
Nachdem die neue Population erzeugt wurde, wird diese durch die Anfangspopulation ersetzt und man führt den Algorithmus erneut aus. 
Dies geschieht so lange, bis man eine hinreichende Lösung für das Problem hat. 




Ich habe meinen Evolutionären Algorithmus so angepasst, dass bei mir ein Induviduum ein Signal ist. 
Ich habe dem Benutzer das Signal mit dem Wearable abspielen lassen und im Anschluss Fragen beantworten lassen. 
Er sollte bewerten wie gut er das Signal erkannt hat. Die Bewertung vom Benutzer war entscheidend um nach der kompletten Bewertung der Population 