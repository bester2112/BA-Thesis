%% taktile geraeate paper.tex
%% $Id: taktilegeraetepaper.tex 28 2007-01-18 16:31:32Z bless $
%%

In dem Paper von XXX \cite{} handelt es um Taktile Displays; was man bei der Erschaffung von Taktilen Display beachten soll und was f{\"u}r Verwendungszwecke es noch gibt. Das Paper ist schon mehr als 15 Jahre alt und weist dennoch auf Informationen hin, die heutzutage noch Relevant sind. Die Inhalte des Papers werden im folgenden beschrieben.
Taktile Displays sind Ger{\"a}te, die dazu benutzt werden um Informationen durch K{\"o}rperkontakt des Menschen {\"u}ber Haptisches Feedback zu {\"u}bermitteln. Diese bilden keinen Konflikt mit den audio-/visuellen Einfl{\"u}ssen. Die Taktilen Displays sind eine Unterst{\"u}tzung der Darstellung der Informationen. 
Beispielsweise kann man Blinden oder Tauben Informationen mittels Taktilen Displays vermitteln, die Sie nicht wahrnehmen k{\"o}nnen.
Dabei werden haptische / sensorische assistive Ger{\"a}te benutzt um die Informationen in der Realen Umgebung in taktile Simulationen umzuwandeln.
Ein wichtiger Aspekt im Paper ist es gewesen, wie man ein solches Taktiles Display entwirft und welche Aspekte man beachten soll. 
Beim Design eines Taktilen Devices sollten leise und leicht, klein sein; wenig Strom verbrauchen; die Taktoren sollten durch Kleidung sp{\"u}rbar sein und am besten so eng wie m{\"o}glich am K{\"o}rper liegen.  
Handys schon damals Vibrationsmotoren gehabt um einem Nutzer darauf Aufmerksam machen wollte, dass man eine Nachricht erhalten hatte.
Die Vibration des Handys war eine Metapher dazu, dass eine andere Person einem an die Schulter r{\"u}tteln w{\"u}rde. \cite{} 
Um es nicht nur Theoretisch zu erkl{\"a}ren hat man ein Taktiles Display als Weste entworfen, bei dem man die Vibrationsmotoren aus Nokia Handys verwendet hat. Mittels der Weste sollte eine Person vom Punkt A zu Punkt B navigiert werden. 
Die {\"u}bermittelten Informationen zur Navigation sind vorw{\"a}rts, zur{\"u}ck, rechts, links, beschleunigen und verlangsamen gewesen. 