%% taktile geraeate paper.tex
%% $Id: taktilegeraetepaper.tex 28 2007-01-18 16:31:32Z bless $
%%

In dem Paper von \cite{gemperle2001design} handelt es um Taktile Displays; 
was man bei der Erschaffung von Taktilen Display beachten soll und was f{\"u}r Verwendungszwecke es noch gibt. 
Das Paper ist schon mehr als 15 Jahre alt und weist dennoch auf Informationen hin, die heutzutage noch Relevant sind. 
Die Inhalte des Papers werden im folgenden beschrieben.
Wie oben bereits erwähnt wurde, sind Taktile Displays Ger{\"a}te, die dazu benutzt werden um Informationen durch K{\"o}rperkontakt des Menschen {\"u}ber Haptisches Feedback zu {\"u}bermitteln. Diese bilden keinen Konflikt mit den audio-/visuellen Einfl{\"u}ssen. 
Die Taktilen Displays sind eine Unterst{\"u}tzung der Darstellung der Informationen. 
Beispielsweise kann man Blinden oder Tauben Informationen mittels Taktilen Displays vermitteln, die Sie nicht wahrnehmen k{\"o}nnen.
Dabei werden haptische / sensorische assistive Ger{\"a}te benutzt um die Informationen aus der realen Umgebung in taktile Simulationen umzuwandeln.
Ein wichtiger Aspekt in diesem Paper ist es gewesen, wie man ein solches Taktiles Display entwirft.
Außerdem gibt es Aspekte die man beim Entwurf beachten sollte:
%Außerdem gibt es Aspekte die man beim Entwurf beachten sollte, wie Stromverbrauch zu reduzieren; leise, leicht und klein sein; sowie das die Taktoren durch die Kleidung sp{\"u}rbar ist und am besten so eng wie m{\"o}glich am K{\"o}rper liegt.
\begin{itemize}
\item leise, leicht und klein sein
\item Reduzierung des Stromverbrauchs
\item die Taktoren sollten durch die Kleidung sp{\"u}rbar sein
\item so eng wie m{\"o}glich am K{\"o}rper liegt
\end{itemize}
%Beim Design eines Taktilen Devices sollten leise, leicht und klein sein; wenig Strom verbrauchen; die Taktoren sollten durch Kleidung sp{\"u}rbar sein und am besten so eng wie m{\"o}glich am K{\"o}rper liegen.  
Seit langem haben Handys bereits Vibrationsmotoren um den Nutzer darauf Aufmerksam zu machen, falls eine Nachricht erhalten wurde.
%Handys haben damals Vibrationsmotoren gehabt um einem Nutzer darauf Aufmerksam machen wollte, dass man eine Nachricht erhalten hatte.
Die Vibration des Handys war eine Metapher dazu, dass eine andere Person einem an der Schulter r{\"u}tteln w{\"u}rde. \cite{gemperle2001design} 
Um es nicht nur Theoretisch zu erkl{\"a}ren hat man ein Taktiles Display als Weste entworfen, bei dem man die Vibrationsmotoren aus Nokia Handys verwendet hat. Mittels der Weste sollte eine Person vom Punkt A zu Punkt B navigiert werden. 
Die {\"u}bermittelten Informationen zur Navigation sind vorw{\"a}rts, zur{\"u}ck, rechts, links, beschleunigen und verlangsamen gewesen. 