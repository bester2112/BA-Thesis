%% Anfroderungen.tex
%% $Id: Anfroderungen.tex 28 2007-01-18 16:31:32Z bless $

Die Aufgabenstellung bestand darin, herauszufinden, ob ein personalisiertes Vibrationsmuster besser als ein vordefiniertes Vibrationsmuster erkannt wird.
Um dies zu loesen musste man sich im Vorfeld ein paar Gedanken ueber die Repraesenatation eines Signals, die Dekodierung und ueber die Uebertragung machen.
Im Folgenden werden diese Entscheidungsfindungen beschrieben.

Die Ue�bertragung

Zu aller erst hat man sich das vorgegebene Armband genauer inspiziert. Dabei konnte man das Armband ueber eine Serielle Schnittstelle sowie ueber Bluetooth Low Energie (LE) nutzen. 
Bei der Nutzung der Seriellen Schnittstelle hatte man zwar den Vorteil, dass man sich nicht mittels Bluetooth auseinander setzen müsse, jedoch ist man an einem Kabel ueber den PC verbunden gewesen, was zur Einschraenkung der Bewegungsfreiheit \dots hat. Um genau diesen Nachteil nicht zu haben hat man sich fuer Bluetooth LE entschieden. Dabei gab es auch einige Nachteile, die man beheben musste. Die Kommunikation mittels Bluetooth LE hatte eine maximale Datenaebertragung von 20 Bytes. Diese wurden jeweils in zwei Bytes Bloecke unterteilt.

Anhand der Begrenzung der Datenuebertragung hat man sich eine geeignete Dekodierung des Signals ueberlegen muessen. 











\dots
Dabei musste man sich definieren wie ein Signal aufgebaut gewesen ist. Dabei musste für ein Signal eine Datenstruktur erstellt werden. 



Dekodierung eines Signals.

Evolutionaerer Algorithmus musste erstellt werden mit allen komponenten, Population Fitness funktion, usw

Kommunikation mit dem Armband musste aufgebaut werden. 

Es sollte eine Studie ausgefuehrt werden um das System zu testen 

Dabei sollte eine Grafische Oberflaeche erstellt werden, damit der Benutzer selbst eine Eingabe in den PC machen konnte um naechste Signale abspielen zu koennen.

Es sollten die Daten ausgewertet werden.
