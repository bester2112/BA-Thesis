%% Anfroderungen.tex
%% $Id: Anfroderungen.tex 28 2007-01-18 16:31:32Z bless $

Die Aufgabenstellung bestand darin, herauszufinden, ob ein personalisiertes Vibrationsmuster besser als ein vordefiniertes Vibrationsmuster erkannt wird.
Um dies zu l{\"o}sen musste man sich im Vorfeld ein paar Gedanken {\"u}ber die Repr{\"a}senatation eines Signals, die Dekodierung und {\"u}ber die {\"U}bertragung machen.
Im Folgenden werden diese Entscheidungsfindungen beschrieben.

Die {\"U}bertragung

Zu aller erst hat man sich das vorgegebene Armband genauer inspiziert. Dabei konnte man das Armband {\"u}ber eine Serielle Schnittstelle sowie {\"u}ber Bluetooth Low Energie (LE) nutzen. 
Bei der Nutzung der Seriellen Schnittstelle hatte man zwar den Vorteil, dass man sich nicht mittels Bluetooth auseinander setzen müsse, jedoch ist man an einem Kabel {\"u}ber den PC verbunden gewesen, was zur Einschr{\"a}nkung der Bewegungsfreiheit \dots hat. Um genau diesen Nachteil nicht zu haben hat man sich f{\"u}r Bluetooth LE entschieden. Dabei gab es auch einige Nachteile, die man beheben musste. Die Kommunikation mittels Bluetooth LE hatte eine maximale Daten{\"a}bertragung von 20 Bytes. Diese wurden jeweils in zwei Bytes Bl{\"o}cke unterteilt.

Anhand der Begrenzung der Daten{\"u}bertragung hat man sich eine geeignete Dekodierung des Signals {\"u}berlegen m{\"u}ssen. 











\dots
Dabei musste man sich definieren wie ein Signal aufgebaut gewesen ist. Dabei musste f{\"u}r ein Signal eine Datenstruktur erstellt werden. 



Dekodierung eines Signals.

Evolution{\"a}rer Algorithmus musste erstellt werden mit allen komponenten, Population Fitness funktion, usw

Kommunikation mit dem Armband musste aufgebaut werden. 

Es sollte eine Studie ausgef{\"u}hrt werden um das System zu testen 

Dabei sollte eine Grafische Oberfl{\"a}che erstellt werden, damit der Benutzer selbst eine Eingabe in den PC machen konnte um n{\"a}chste Signale abspielen zu k{\"o}nnen.

Es sollten die Daten ausgewertet werden.
