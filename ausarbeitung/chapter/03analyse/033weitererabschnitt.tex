%% Weiterer Abschnitt.tex
%% $Id: WeitererAbschnitt.tex 28 2007-01-18 16:31:32Z bless $

% System aufbau / ausf{\"u}hrung der Studie / analyse aufbau
% beschreibt, was ich gemacht habe.

\paragraph{R{\"u}ttelflug}

%Das vorgegebene Wearable von TECO ist ein {\"a}hnliches  
Aus dem Paper von \cite{pescara2016ruttelflug} wurde ein Wearable f{\"u}r das Paragliding entworfen.
Beim Entwurf des Wearables hat man sich an den vorhin genannten Anforderungen von \cite{gemperle2001design} gehalten.
Das Wearable besitzt einen Traktor oberhalb und einen unterhalb Handgelenks und wurde sowohl in einer Laborstudie als auch in einer Feldstudie getestet.
%Die daraus ergebenen Ergebnissen  
F{\"u}r die Laborstudie wurden Vibrationssignale in der L{\"a}nge von 100, 200, 400, und 800ms abgespielt. 
Jeder Proband hat dieselbe Reihenfolge von Aufgaben ausgef{\"u}hrt bekommen. 
Eine der Aufgaben der Probanden war den Namen und die L{\"a}nge des Signals zuzuordnen. 
Aus der Laborstudie hat sich ergeben, dass die Probanden die Signall{\"a}nge 800ms mit einer Genauigkeit von 97,6\% sowie die beiden Singnalsl{\"a}nge 200 und 400ms zu 100\% richtig erkannt haben. 

Die generischen Werte f{\"u}r die Erkennung der Muster hat man aus dem Paper {\"u}bernommen.