%% eval.tex
%% $Id: eval.tex 5 2005-10-10 20:55:48Z bless $

\chapter{Evaluierung}
\label{ch:Evaluierung}
%% ==============================
Hier kommt der Nachweis, dass das in Kapitel~\ref{ch:Entwurf}
entworfene Konzept auch funktioniert. Leistungsmessungen einer
Implementierung werden auch immer gerne gesehen.



%% ==============================
\section{Initialisierung der Grenzen}
%% ==============================
\label{ch:Evaluierung:sec:Abschnitt1}



%% ==============================
\section{Abschnitt 2}
%% ==============================
\label{ch:Evaluierung:sec:Abschnitt2}



%% ==============================
\section{Zusammenfassung}
%% ==============================
\label{ch:Evaluierung:sec:zusammenfassung}

Am Ende sollten ggf. die wichtigsten Ergebnisse nochmal in \emph{einem}
kurzen Absatz zusammengefasst werden.

%%% Local Variables: 
%%% mode: latex
%%% TeX-master: "diplarb"
%%% End: 
